\lecture{10}
\begin{example}
    \begin{enumerate}
        \item \(\{1, -1, 0, 1, -1, 0, 1, -1, 0, \cdots\}\)\\
        \[\displaystyle\limsup_{n\to\infty}a_n = 1 \quad \displaystyle\liminf_{n\to\infty}a_n = -1\]
        Indeed, \(X_n = 1 ~ \forall ~ n ~\in~ \mathbb N\), \(Y_n = -1~ \forall ~ n ~\in~ \mathbb N\)
        \item 
        \begin{align*}
            a_n &= 1 + \frac{1}{n}, ~\forall~ n ~\in~\mathbb~ \forall ~ n ~\in~ \mathbb N\\
            a_n &= \{2, 1~\frac{1}{2},1~\frac{1}{3}, 1~\frac{1}{4}, \cdots\}\\
            \left(X_n\right)_{n=1}^\infty &= \{2, 1~\frac{1}{2}, 1~\frac{1}{3}, \cdots\}\\
            \distplaystyle\limsup_{n\to\infty} a_n  &= \displaystyle\lim_{n\to\infty} X_n = 1\\
            \text{Now, }~ \left(Y_n\right)_{n=1}^\infty &= \{1, 1, 1, \cdots\}\\
            \distplaystyle\liminf_{n\to\infty} a_n  &=  1\\
        \end{align*}
    \end{enumerate}
\end{example}

\section{Subsequences}
\begin{definition}
    Let \(\left(a_n\right)_{n=1}^\infty\) be a sequence and \(\{n_1, n_2, n_3, \cdots\}\) be a strictly increasing sequence of natural numbers.\\
    Then, \(\left(a_{n_k}\right)_{k=1}^\infty\) is called a \underline{subsequence} of \(\left(a_n\right)_{n=1}^\infty\).
\end{definition}
Add in tex diagram here of a sequence and then choosing certain terms of the sequence, creating a subsequence. \(a_3 \to n_1, a_4 \to n_2, a_6 \to n_3, a_8 \to n_4, \cdots\)\\
In this example, \(n_1 = 3, n_2 = 4, n_3 = 6, \cdots\)\\
The subsequence is \(\{a_3, a_4, a_6, a_8, \cdots\}\)\\
\begin{example}
    \(\left(a_n\right)_{n=1}^\infty = \{1, -1, 1, -1, 1, -1, \cdots\}\)\\
    Some subsequences:\\
    \begin{align*}
        \{a_1, a_3, a_5, \cdots\} &= \{1, 1, 1, \cdots\}\\
        \{a_2, a_4, a_6, \cdots\} &= \{-1, -1, -1, \cdots\}\\
    \end{align*}
\end{example}
\begin{proposition}
    \(\left(a_n\right)_{n=1}^\infty\) diverges.\\
\end{proposition}
\begin{proof}
    Assume \(\displaystyle\lim_{n\to\infty} a_n = \alpha\).\\
    There are three possibilities: \(\alpha = 1\) or \(\alpha = -1\) or \(\alpha \neq \pm 1\).\\
    Assume \(\alpha \neq \pm 1\) (The other cases are similar).\\
    For every \(\varepsilon > 0\), there exists \(N \in \mathbb N\) such that if \(n \ge N\), then \(|a_n - \alpha| < \varepsilon\).\\
    Insert Diagram here of some epsilon neighbourhood around a limit.\\
    Take \(\varepsilon = \min\{\frac{|\alpha - 1|}{2}, \frac{|\alpha - (-1)|}{2}\}\).\\
    If \(a_n = 1\), then \[|\alpha - a_n| = |\alpha - 1| > \frac{|\alpha - 1|}{2} \ge \varepsilon\]
    If \(a_n = -1\), then 
    \[|a_n - \alpha| = |-1 - \alpha| = |\alpha - (-1)| > \frac{|\alpha - (-1)|}{2} \ge \varepsilon\]
    Thus, \(|a_n - \alpha| > \varepsilon\) for all \(n \in \mathbb N\).\\
    This contradicts the assumption that \(|a_n - \alpha| < \varepsilon\) for all \(n \in \mathbb N\).\\
\end{proof}
\begin{definition}
    The number \(a \in \mathbb R\) is a cluster point of \(\left(a_n\right)_{n=1}^\infty\) if for every \(\varepsilon > 0\), there exists infinitely many \(n \in \mathbb N\) such that \(|a_n - a| < \varepsilon\)
\end{definition}
Insert diagram here of an epsilon neighbourhood around a few clustered points and the limit.\\
\begin{note}
    \(\displaystyle\limsup a_n\) and \(\displaystyle\liminf a_n\) are the greatest and the smallest cluster points respectively.
\end{note}
\begin{theorem}
    Let \(\left(a_n\right)_{n=1}^\infty\) be a sequence. Take \(a \ in \mathbb R\).\\
    \begin{enumerate}
        \item a is a cluster point if and only if for every \(\varepsilon > 0\) and every \(N \in \mathbb N\), there exists \(n \ge N\) such that \(|a_n - a| < \varepsilon\)\\
        \item a is a clusterpoint if and only if \(\left(a_n\right)_{n=1}^\infty\) has a subsequence \(\left(a_{n_k}\right)_{k=1}^\infty\) such that \(\displaystyle\lim_{k\to \infty} a_{n_k} = a\).\\
        \item \(\displaystyle\lim_{n\to\infty} a_n = a\) if and only if every subsequence of \(\left(a_n\right)_{n=1}^\infty\) converges to a.\\
    \end{enumerate}
\end{theorem}
\begin{proof}\\
    \begin{enumerate}
        \item 
    \(\left(\implies\right)\)\\
    Let a be a cluster point. Take \(\varepsilon > 0\) and \(N \in \mathbb N\). By definition, there are infinitely many \(n\in \mathbb N\) such that \(|a_n - a| < \varepsilon\). Obviously, some of these \(n\) must be greater than \(N\).\\
    \(\left(\impliedby\right)\)\\
    Fix \(\varepsilon > 0\). We want to show there are infinitely many \(n\) such that \(|a_n - a| < \varepsilon\). \\
    Applying the condition with \(N = 1\), we find that there is \(n_1 \in \mathbb N\) such that \(|a_n - a| < \varepsilon\).\\
    Applying the condition again, this time with \(N = n_1 + 1\): There exists \(n_2 \ge N = n_1 + 1 > n_1\) such that \(\big\vert a_{n_2} - a\big\vert < \varepsilon\).\\
    Apply the condition again, this time with \(N = n_2 + 1\).\\
    Continuing with this process, we obtain a subsequence \(\left(a_{n_k}\right)_{k=1}^\infty\) such that \(\big\vert a_{n_k} - a\big\vert < \varepsilon\) forf all \(k \in \mathbb N\). Thus, \(a\) is a cluster point.
\end{enumerate}
\end{proof}
